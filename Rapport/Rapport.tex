\documentclass[12pt]{article}

\usepackage[T1]{fontenc} 
\usepackage[utf8]{inputenc}
\usepackage[francais]{babel}

\title{{\bf Rapport de projet} \\ \normalsize{Environnement de Programmation}}
\date{}
\author{\normalsize{BONNET \bsc{Thomas} , JAJOUX \bsc{Jeremy}}}

\begin{document}

\maketitle

\section{Fonctions de base du jeu
\\{\small Soit les fonctions de grid.c}}

Ici la plupart des fonctions étaient simples à programmer car elles sont de constructions moindres et ont un objectif unique et court.\\

Les fonctions ayant le plus posé de problèmes furent les fonctions relatives aux mouvement des tuiles, c'est à dire ``can\_move'' et ``do\_move'', car ces dernières nécessitent de prendre en compte toutes les différentes situations possibles lors d'un mouvement ou d'une fusion de tuile.\\

C'est pourquoi, lors des premières ébauches du code, toutes ces fnctions étaient très longues et répétitives afin d'inclure toutes les possibilité sans se soucier de la structures de celles-ci.
L'étape suivante à donc consistée à réduire la quantité de code et à créer des fonctions annexes, comme ``move'' ou ``tile\_fusion'', pour alléger le code de chaque fonction et permettre une disparition des répétitions avec des sous-fonction avec différents arguments selon les situation plutôt qu'une section de code répétée.

\section{Fonctions pour faire fonctionner le jeu
\\{\small Les fonctions contenues dans play.c}}


\section{Les Tests
\\{\small Trouvés dans test.c}}

\section{Le Makefile}

\section{Les Stratégies}

\section{Interface Graphique
\\{\small Le corps du fichier 2048.c}}

%\begin{verbatim}
%Option utile pour écrire du code
%\end{verbatim}

\end{document}
